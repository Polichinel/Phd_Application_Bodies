%\pagestyle{article}
\documentclass[a4paper]{article}
\usepackage[english]{babel}
\usepackage[utf8]{inputenc}
\usepackage{graphicx} % for figures
\usepackage[section]{placeins} %This prevents placing floats before a section.
\usepackage{csquotes}

\usepackage{wrapfig} % curtesy of LAURA PETRA - til dit billede i CV

% \usepackage{marginnote} % For margin years CV
% \newcommand{\years}[1]{\marginnote{\scriptsize #1}} % New command for including margin years
% \renewcommand*{\raggedleftmarginnote}{}
% \setlength{\marginparsep}{7pt} % Slightly increase the distance of the margin years from the contant
% \reversemarginpar

\usepackage{natbib}% better citation
%\usepackage{hyperref} %autoref
\usepackage[hidelinks]{hyperref} %hidelinks to remove ugly blue format
\usepackage{amsmath} % for \tag and \eqref macros in mathematical eq.

%% Sets linestretch, paragraphstrech, indentation and footnote stuff 
\usepackage{parskip} %space between paragraphs
\parskip=0pt %set space between paragraphs
\setlength\parindent{12pt} %paragraf indentation
% \usepackage[onehalfspacing]{setspace} %linespacing; does not affect footnotes
\usepackage[doublespacing]{setspace} %linespacing; does not affect footnotes

\setlength{\footnotesep}{0.7\baselineskip}% space between footnotes
\usepackage[hang,flushmargin]{footmisc} %% removes identation in footnoteshttps://www.overleaf.com/project/5b98e00a21d3bd15ac5a2e86

\usepackage{makecell} % make cells in tabels for longer text

\usepackage[colorinlistoftodos]{todonotes}

%% For header and footer (1)
% Marco
\usepackage{fancyhdr}
\pagestyle{fancy}
\textwidth = 452pt% 424pt % test width ish
\oddsidemargin = 0pt %18pt % margin width ish

\fancyheadoffset{0 in} % Shifty solutions..

\fancyhf{} %% clear defuelt header and footer

%% For header and footer (2)
%Specifics
\lhead{Simon P. von der Maase}
\rhead{\today}

\lfoot{University of Copenhagen}
\rfoot{\thepage}
\renewcommand{\footrulewidth}{0.8pt}

\title{\textbf{Bodies as Battleground} \\ PhD Application}

\author{Simon Polichinel von der Maase}
\date{July 2019}

\begin{document}

	\begin{titlepage}
		\maketitle
		%Character count: 43.950/44.000\\
		\noindent\rule{\linewidth}{0.4pt}
		\begin{figure}[h]
			\centering
			\includegraphics[scale=0.32]{KU_logo.png}
		\end{figure}
		\thispagestyle{empty} % removes page number on front page
	\end{titlepage}
    \tableofcontents
\pagebreak

% \section{Application} if below should be subsections.

\section{Motivation and background} 
%Cover Letter detailing your motivation and background for applying for the specific PhD project

% "Are you the right candidate? Preferred applicants will have an interest in questions related to conflict, gender, images, photojournalism and quantitative methodologies. We are looking for candidates with strong analytical skills and proven quantitative methodological qualifications. Experience in working with the Georeferenced Event Dataset from the Uppsala Conflict Data Program or similar datasets is an advantage but not a requirement."

% Du kan overveje at sætte det således op....
%1)interest in questions related to conflict, gender, images, photojournalism and quantitative methodologies. '
%2)We are looking for candidates with strong analytical skills and proven quantitative methodological qualifications. 
%3)Experience in working with the Georeferenced Event Dataset from the Uppsala Conflict Data Program or similar datasets is an advantage but not a requirement."

% intro
As a political scientist with specialization in advanced quantitative methodologies and conflict studies, I was immediately intrigued when I learned of your project. Indeed I see a clear match between my interests, knowledge and skills on one side and your project on the other.\par

% Why you apply for this?
Firstly, conflict research has been my prime politological subject of interest since I started my bachelor and one of two focal points during my Bachelor's and Master's program. Secondly, my other focal point has been advanced quantitative methods. Indeed, the intersection between computational methods and conflict studies has been the fulcrum of my Master's program. Thirdly, I believe the power to control women's bodies to be a central part of any conflict. As such, I see great potential and importance in pursuing a gender focused approach to conflict studies. Lastly, I have long wanted to use image data in my research, but have struggled to find any data of relevance -- until you presented your project, that is. The data you have collected is, in my humble opinion, nothing less than a potential treasure trove of information. As such, my interests align perfectly with the subject and approach of your project.\par

% What have you done before?
Appropriately, my interests are matched by my abilities. Regarding politological subjects I have worked extensively with intra-state conflicts, civil wars and democratic breakdowns. I also have experience with theories on peace-building and post-conflict reconciliation. Concerning quantitative subjects I have worked with Frequentist econometrics, Bayesian statistics, Machine/Deep Learning, geo-spatial analysis, panel data, web-scraping, text analysis, Social Network Analysis, data visualization and Data Science in general. I have acquired my skills and knowledge through courses from the Department of Political Science (UCPH), the Department of Economics (UCPH), the Center of Social Data Science (UCPH) and the Department of Applied Mathematics and Computer Science (DTU) - along with countless hours spend online, looking for more knowledge, new skills and better solutions. As such, my theoretical, analytical and methodological foundations are strong and well suited for your project.\par % More pol or less quant? 

% more specifically assignment and master thesis
I have written both my master's thesis \citep{SPECIALE} and a free-assignment \citep{Maase} on the subject of advanced quantitative methods, geo-spatial data and conflict studies, specifically using the UCDP database. In both of these efforts I employ highly disaggregated geo-spatial data (from the UCDP and the PRIO grid) in tandem with advanced Machine learning techniques to predict conflict on sub-national level \citep{Maase, SPECIALE}. As such, I am confident that I can handle the challenges of your project and bring both relevant insights and experience to the effort.\par


\section{Project description}
%(max. 5 pp. double-spaced, not including bibliogs raphy or time schedule) 

% To outline how I will approach the project I will first summarize what I perceive as the problem we try to address. Then I will outline the challenges we face addressing said problem. Lastly, I will present my proposed solutions to these challenges.\par %, effectively outlining the blueprint for my project in the process.\par

% \subsection{The problem}

% ++++++++++++++++++++++++++++++++++++++++++++++++++++++++++++++++++++++++++++++++++++++++++
% THE WORLD; what is the problem?
% ++++++++++++++++++++++++++++++++++++++++++++++++++++++++++++++++++++++++++++++++++++++++++

% Intro 
It has been argued that as violent conflict increases, gender roles change towards more traditional norms. Men are seen as providers of protection and women being in need of this protection \citep{elshtain1995women, carpenter2003women}. It has also been argued that women are perceived as more emotional while men are perceived as more rational doing conflicts \citep{elshtain1993public}. If this is indeed the case then it appears conflict transforms women from actors into subordinate victims -- at least in the eyes of the beholder.\par 

% The more specific problem
This constitutes a number of problems. Firstly if violent conflict in itself diminishes the roles of women to that of emotional victims they might lose their voice and agency during such conflicts \citep{hansen2000gender, hansen2000little}. Naturally this leaves women in an unfavorable position \citep[294-297]{hansen2000little} -- a position which might linger long after the conflict has settled. Secondly, by reducing the role of women to fragile beings in need of male protection, we will also fail to recognize the very active -- and sometimes destructive -- role women have played in conflicts \citep[66]{hansen2000gender}. Such blind-spots will impede reconciliation and obscure valuable lesson which could otherwise be used for future conflict prevention.\par

% Why this is very important:
As such, the question of how violent conflicts interact and change the perception of gender is a highly relevant question not just of academic interest, but also of practical importance. Given the data obtained for this project we can now begin to answer such questions by examining the relationship between the level of violence in armed conflict and the representation of gender in war photography.\par

% there is "[...] a correlation between the level of violence in armed conflict and the representation of gender in war photography?"

% to what extent transformations of gender roles are apparent in the photographs taken by journalists in conflict zones.\par

% Bridge
Naturally, the project also faces a number of challenges. The use of geo-referenced disaggregated data in conflict studies is still in its infancy, and the addition of image-data even more so. Further more, past quantitative efforts in conflict studies are experiencing increasing criticism for using a fundamentally flawed approach to evaluating statistical results. As such, a myriad of methodological and theoretical challenges lies before us -- but so does a lot of opportunities.\par


% ++++++++++++++++++++++++++++++++++++++++++++++++++++++++++++++++++++++++++++++++++++++++++
% The Science; what have been done before?
% ++++++++++++++++++++++++++++++++++++++++++++++++++++++++++++++++++++++++++++++++++++++++++
\subsection{The challenges}

%images -- inhibits the questions we can answer...
Importantly, the fact that we are working with journalistic photographs determines the kind of questions we can ask. We can only ask questions regarding how the depiction of gender roles in conflict zones correlates with the level of violence -- not about how the actual gender roles or norms might change in the area. Furthermore, we most consider the possibility that there is a difference between the photographs taken, the photographs send to the editor and the photographs published. In the same manner we must also examine whether the gender of the journalists themselves matters. These er all potential modifiers, confounders and interactions to be accounted for. Understanding that the data reflects depiction and presentation by observers rather than experiences by subjects is crucial when we formulate our hypothesis, when we construct our models and label (code) the images.\par

%images -- Challenges of how to classify the images
The labeling of the images itself also constitutes a huge practical challenge. In practise we do not want to classify whole pictures but rather specific objects, persons or situations in the pictures. Adding the scope of the dataset to this, we would be hard pressed to label all images included in the data set. We could settle for a subset, but given the rapid development in computational techniques for image recognition \citep{williams2019images, francois2017deep}, this would constitute a missed opportunity. As such, we must consider how to use these advanced new tools to effectuate the classification processes.\par 

% Why disaggragated data
Having classified the images, we must also aggregate them to some unit of analysis. It is important to recognize that time and space are, fundamentally, continues dimensions. However, to work with these dimensions using quantitative tools we need to aggregate them at some discrete level. Most traditional quantitative studies of conflicts have aggregated these dimensions into \emph{country-years} (see such articles as \cite{Collier_Hoeffler_1998, Fearon_Laitin_2003, Collier_Hoeffler_2004, Fearon_2004, Ross_2004, Fearon_2005, Hegre_Sambanis_2006, Goldstone_2010}). While this aggregation might be appropriate in some settings, it is a very coarse unit of analysis. In reality, conflicts -- especially internal -- rarely encompasses entire countries, but are often confined to specific regions \citep[487]{Cederman_Gleditsch_2009}. As such, we miss important nuances and patterns when treating conflict as a country-wide phenomenon. Some regional conflicts will be presented as country-wide, while others will be treated as non-conflicts. As a result past quantitative analyses have often missed important aspects and mechanisms of conflicts \citep{Cederman_Gleditsch_2009, Cederman_Gleditsch_Buhaug_2013}. As formulated by \cite{Cederman_Gleditsch_2009}: "If our theories are disaggregated, then our empirical analyses and research designs should reflect this" \citep[490]{Cederman_Gleditsch_2009}. A point which is naturally as relevant for the temporal dimension of conflict as it is for the spatial dimension.\par 

% The challenges of disaagregated data :
Given that the dataset contains images with a approximate location and time stamp, and that we will use the UCDP to track violence in conflicts, we have a great opportunity to create the specific unit of analysis which best aligns with our theoretical expectation regarding the relationship between the depiction of gender roles and violent conflicts. This is crucial, because if we are to ask and answer the right questions we need a theoretical sound unit of analysis, we need to consider our various hypotheses when we create this unit and we need to consider our unit when we handle our data, when we construct our models and when we evaluate our results.\par 

% An example (autocorrolation)
Specifically, using a highly disaggregated unit of analysis increases the relevance of handling both temporal and spatial auto-correlation. As an example, if a violent conflict is present in one village, one specific day it stands to reason that said village will still experience the fallout the next day, the next week and probably also the next month -- if not even longer depending on the magnitude of the violence \cite[15-17]{SPECIALE}. In similar manner a neighboring village might also experience some fallout even if it was not the specific geographical site of the violence. If a journalist where to take a picture the week before the act of violence, and the week after the act of violence, we would not expect these two pictures to be similar, even if they where both taken in the absence of actual violence. The same might be true for pictures taken in the neighboring village. As presented in \cite{SPECIALE} past efforts have tried to ameliorate such problems in various ways, but they have largely failed to create a methodological and theoretical coherent solution for highly disaggregated data.\par 

% The problem of evaluation
Lastly, we must evaluate our results. Traditionally, researchers have used statistical tools to estimate the relationship between various features and different aspects of conflict \citep[8]{chadefaux2017conflict}. A central element of this approach has been to estimate whether said relationship could be considered \emph{statistically significant} \citep[363-364]{Ward_Greenhill_Bakke_2010}. However, using statistical significance as the prime metric of evaluation is now considered imprudent \citep{Ward_Greenhill_Bakke_2010, Schrodt_2014, chadefaux2017conflict}. Indeed, the use of significance testing in conflict studies has been consistently criticized for almost two decades \citep{king_zeng_2001b, Ward_Greenhill_Bakke_2010, Goldstone_2010, Schrodt_2014, chadefaux2017conflict}. The addition of other evaluations regimes such as \emph{out-of-sample predictions}, \emph{k-fold cross-validation}, \emph{leave-one-out cross-validation} and various \emph{information criteria} are now highly encourage, no matter whether the specific goal of a study is explanation or prediction \citep{Ward_Greenhill_Bakke_2010, Schrodt_2014, Mcelreath_2018}. This project will focus on explanation. To facilitate correct interpretations and evaluations of our estimations and results we most adhere to the criticism of past efforts.\par 

% bridge
If we are to truly take advantage of the data available for this project and hope to create meaningful and robust insights into the dynamics between gender depiction and violent conflict we most take these challenges very seriously. Intriguingly, solving such challenges will also allow us to create micro level insights regarding not only the correlation between violent conflicts and gender perception but also more specifically examine how gender perceptions diffuse from violent conflicts out through time and space.\par

% ++++++++++++++++++++++++++++++++++++++++++++++++++++++++++++++++++++++++++++++++++++++++++
% This project; How are you gonna do it?
% ++++++++++++++++++++++++++++++++++++++++++++++++++++++++++++++++++++++++++++++++++++++++++

\subsection{The solutions}

%Formulating the hypothesis
The first step will be to formulate clear and precise theoretical expectations regarding our subject of analysis \cite[30-53]{Cederman_Gleditsch_Buhaug_2013}. As presented in the scholarship-announcement, the overarching research question for the specific sub-project concerns whether there is "[...] a correlation between the level of violence in armed conflict and the representation of gender in war photography?" \citep{bodies}. Given our theoretical foundations we might expect women in violent conflict zones to be depicted in more conservative and emotional settings. As violence increases, we might expect women to be depicted in more traditional and more concealing clothes. We might expect them to be less visible in imagery of public places and we might expect them the be depicted as in need of (male) projection. Conversely we might expect men to be depicted as protectors. We might also expect men to be depicted in situations of restraint of reason. 
Rigorously consulting the relevant theoretical insight of past efforts I will formulate new and concise hypotheses regarding the relationship between gender depiction and conflict violence.\par 

% Bayesian hierarchical models for meta hypotheses
I will also raise questions regarding whether the gender of the journalist and the fate of the photography after it was captured is related to how gender is depicted. Such questions also needs to be formulated as hypotheses, but these will take the form of more conditional hypothesis including various forms of interactions. To capture such conditions and interactions I will use Bayesian hierarchical models \citep{Gelman_2006, Gelman_2013, Mcelreath_2018}. Such models are primed for more complex hypothesis and utilizes the information contained in the data in a much more effective way then more conventional econometric models \cite[355]{Mcelreath_2018}. Using Bayesian hierarchical models will also help me battle both underfitting and overfitting and it will help me handle the imbalance (most places does not experience violence most of the time) of the data \cite[356-357]{Mcelreath_2018}. \par

As these models are Bayesian I can readily employ Leave-one-out cross-validation (LOO) and the Watanabe-Akaike Information Criteria (WAIC) along traditional significance testing for evaluation. This facilitates accurate model comparison and deeper insights into our models \cite[165-205]{Mcelreath_2018} \par

% -> fortæl hvordna du vil classificere billede jf. de hypoteser 
Having precise hypotheses and using the appropriate models, however, is not enough to insure viable results. Our models will never be better then the data we feed them. As such, the photographs must be coded concise and in a manner which corresponds to the hypotheses we want to test. If one hypothesis concerns that women will wear more traditional cloths, we will need to conceptualize what this means, whether this should be a binary or interval based measure and so forth. Having conceptualized clear coding rules capturing all relevant information in the photographs, I will label a large subset of pictures accordingly. To take full advantage of the dataset I will furthermore train a \emph{Convoluted Neural Network} to classify the remaining photographs \citep[120-122]{francois2017deep}. Not only will this ease the burden of labeling all the pictures, it will also enable us to automatically label new images for future endeavours.\par    

%  Aggregating images and violence at the level of PRIO grid.
Having classified all images, these need to merge with the data on conflict fatalities from the \emph{Uppsala Conflict Data Program} (UCDP) \citep{Sundberg_2013, Croicu_Sundberg_2017, UCDP_2017}. This data contains records of conflict fatalities and the corresponding coordinates and dates. Naturally we cannot match pictures and episodes of conflict violence on the specific coordinates and time. We need some manageable geographical and temporal unit of analysis. The specifically temporal unites will depend on how precise the time-stamp on the photographs are. As basis for the geographical unit I will use the PRIO grid cells. The PRIO grid is a global grid dividing the world -- excluding Greenland and Antarctica -- into grid cells of $0.5 \times 0.5$ decimal degrees, which corresponds to roughly $50km\times50km$ squares at the equator \citep[367]{Tollefsen_2012}. This resolution should be sufficiently fine to capture within country variation while still keeping the data-handling manageable. Importantly, compared to other sub-national units such as provinces, counties and municipalities the grid cells are stationary over time, they all have similar shapes and are largely the same size across unites. The shapes are also completely exogenous to any political, cultural or social feature under investigation \citep[356]{Tollefsen_2012}. As such, the images and the data on conflict violence will be aggregated it the geographical level of the PRIO grid cells and some not-yet-specified sub-yearly temporal unit.\par

% Handling autocorrolations with GP's in BMML's
Using such highly disaggregated unit of analysis, I must handle the massive temporal and spatial auto-correlation between units. For this I will use Gaussian processes \citep{williams2006gaussian}. This advanced machine learning technique is remarkable will suited of such challenges \citep{Gelman_2013, gelfand2016spatial, Mcelreath_2018, SPECIALE}. I illustrated the potential of this approach in \cite{SPECIALE} where I also explain why Gaussian process fits with the theoretical expectations regarding various conflict patterns\citep[22-27]{SPECIALE}. Enquiringly, Gaussian processes can readily be incorporated in Bayesian hierarchical models \citep[410-419]{Mcelreath_2018}. Thus Gaussian process will allow me to handle the auto-correlation of disaggregated data.\par

% EVALUTION!

% Final
Your project is highly relevant and your data intriguing. But it is also a challenging and complex project. As illustrated above I have a both the motivations, the skills and also a very concrete plan to tackle such project. I offer you cutting edge knowledge on both conflict studies and quatitative methodes, In return I hope you will consider my application to be part of your project: Bodies as Battlegrounds.\par
% \hfill\par
% \hfill\emph{Simon Polichinel von der Maase}


%-> så fortæl dem hvad svaret på projectet bliver (well elelr hvad du tror.)

\pagebreak

\section{Time schedule} 

\begin{center}
\begin{tabular}{ m{7.5cm}| m{7.5cm} } 
    \textbf{Autumn/Winter}              & \textbf{Spring/Summer}\\
	\hline
    \textbf{2019} Survey and study the litterateur on gender perception, presentation and depiction in conflict zones to create theoretical grounded and thoroughly thought out hypothesis.
    & \textbf{2020} Preprocessing the image data and (hand) labeling of a representative subset.\\
    
    \hline
    \textbf{2020} Create, train and execute the Convoluted network for classification of all images not labeled.
    & \textbf{2021} Construct the Bayesian hierarchical models including the Gaussian processes in accordance with the formulated hypotheses\\
    
    \hline
    \textbf{2021} Running and interpreting all hierarchical models.
    & \textbf{2022} Finish Ph.D. project and convey/present final results.\\	

%\hline
\end{tabular}
\end{center}
\pagebreak

\section{Temporary bibliography}
% - Temporary
\bibliographystyle{apalike} 
\bibliography{app_conf.bib}

\pagebreak
\section{Appendix}

\subsection{Budget}

\begin{center}
\begin{tabular}{  m{10cm} m{4cm} } 

	\hline
	\textbf{Activity}    & \textbf{Expenditure}\\
	\hline
	    &                               \\
	Conferences attendances (at least two)   & 6000 - 16.000 Dkr.                       \\
    Research stay at Oslo, Uppsala or Columbia (travel ex.)  & 1000 - 10.000 Dkr.				\\
	Cloud computing (rent)   & 2000 - 4000 Dkr.			                                \\
    Course at DTU (at least two) & 0 Dkr.			                                \\
% 	Activity 5   & X Dkr.	            \\	
%     Activity 6   & X Dkr.			    \\
%     Activity 7   & X Dkr.			    \\
% 	Activity 8   & X Dkr.			    \\
%  	Activity 9   & X Dkr.				\\
 	    &                               \\
 	\hline
    \textbf{Total}       & 9000 - 30.000 Dkr.			    \\
    \hline

\end{tabular}
\end{center}


\pagebreak
\subsection{CV}

My CV can be found in the appended file "\textbf{CV\_bodies.pdf}"

%\pagebreak
\subsection{Bachelor's Diploma}

My bachelor's diploma can be found in the appended file "\textbf{Bachelors\_diploma.pdf}"

\subsection{The master's thesis contract and approval}

The master's thesis contract can be found in the appended file "\textbf{Masters\_thesis\_contract.pdf}"\par

The corresponding approval can be found in the appended file "\textbf{Masters\_thesis\_approval.pdf}"

\subsection{Grades}

At the time of writing my average at this master's program is 10.8. I finished my bachelor's degree with an average of 7.5. All grades (from master's and bachelor's alike) can be found in the appended file "\textbf{All\_grades.pdf}". The official bachelor-grade-card can be found in the append file "\textbf{Bachelors\_grade\_card.pdf}".\par


\end{document}
