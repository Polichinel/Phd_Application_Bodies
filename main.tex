%\pagestyle{article}
\documentclass[a4paper]{article}
\usepackage[english]{babel}
\usepackage[utf8]{inputenc}
\usepackage{graphicx} % for figures
\usepackage[section]{placeins} %This prevents placing floats before a section.
\usepackage{csquotes}

\usepackage{wrapfig} % curtesy of LAURA PETRA - til dit billede i CV

% \usepackage{marginnote} % For margin years CV
% \newcommand{\years}[1]{\marginnote{\scriptsize #1}} % New command for including margin years
% \renewcommand*{\raggedleftmarginnote}{}
% \setlength{\marginparsep}{7pt} % Slightly increase the distance of the margin years from the contant
% \reversemarginpar

\usepackage{natbib}% better citation
%\usepackage{hyperref} %autoref
\usepackage[hidelinks]{hyperref} %hidelinks to remove ugly blue format
\usepackage{amsmath} % for \tag and \eqref macros in mathematical eq.

%% Sets linestretch, paragraphstrech, indentation and footnote stuff 
\usepackage{parskip} %space between paragraphs
\parskip=0pt %set space between paragraphs
\setlength\parindent{12pt} %paragraf indentation
\usepackage[onehalfspacing]{setspace} %linespacing; does not affect footnotes
\setlength{\footnotesep}{0.7\baselineskip}% space between footnotes
\usepackage[hang,flushmargin]{footmisc} %% removes identation in footnoteshttps://www.overleaf.com/project/5b98e00a21d3bd15ac5a2e86

\usepackage{makecell} % make cells in tabels for longer text

\usepackage[colorinlistoftodos]{todonotes}

%% For header and footer (1)
% Marco
\usepackage{fancyhdr}
\pagestyle{fancy}
\textwidth = 452pt% 424pt % test width ish
\oddsidemargin = 0pt %18pt % margin width ish

\fancyheadoffset{0 in} % Shifty solutions..

\fancyhf{} %% clear defuelt header and footer

%% For header and footer (2)
%Specifics
\lhead{Simon P. von der Maase}
\rhead{\today}

\lfoot{University of Copenhagen}
\rfoot{\thepage}
\renewcommand{\footrulewidth}{0.8pt}

\title{\textbf{Bodies as Battleground} \\ PhD Application}

\author{Simon Polichinel von der Maase}
\date{May 2019}

\begin{document}

	\begin{titlepage}
		\maketitle
		%Character count: 43.950/44.000\\
		\noindent\rule{\linewidth}{0.4pt}
		\begin{figure}[h]
			\centering
			\includegraphics[scale=0.32]{KU_logo.png}
		\end{figure}
		\thispagestyle{empty} % removes page number on front page
	\end{titlepage}
    \tableofcontents
\pagebreak

% \section{Application} if below should be subsections.

\section{Motivation and background} 
%Cover Letter detailing your motivation and background for applying for the specific PhD project

% "Are you the right candidate? Preferred applicants will have an interest in questions related to conflict, gender, images, photojournalism and quantitative methodologies. We are looking for candidates with strong analytical skills and proven quantitative methodological qualifications. Experience in working with the Georeferenced Event Dataset from the Uppsala Conflict Data Program or similar datasets is an advantage but not a requirement."

% Du kan overveje at sætte det således op....
%1)interest in questions related to conflict, gender, images, photojournalism and quantitative methodologies. '
%2)We are looking for candidates with strong analytical skills and proven quantitative methodological qualifications. 
%3)Experience in working with the Georeferenced Event Dataset from the Uppsala Conflict Data Program or similar datasets is an advantage but not a requirement."

% intro
As a political scientist with specialization in advanced quantitative methodologies and conflict studies, I was immediately intrigued when I learned of your project. Indeed I see clear match between my interest, knowledge and skills on one side an your project on the other. To support this assertion, I here outline my academic background and motivation for applying for this scholarship.\par

% Why you apply for this?
First of all, conflict research has been my prime politological subject of interest since I started my bachelor and it has been one of two focal point doing my Bachelor's and Master's program. Secondly, the other focal point of my studies have been advanced quantitative and computational methods. Indeed, the intersection between computational methods and conflict studies has been the fulcrum throughout my masters program. Thirdly, I believe the power to control woman's bodies to be a central part of any conflict. As such, I see great potential and importance in pursuing a gender focused approach to conflict studies. Lastly, I have long wanted to use image data in my research, but have struggled to find any data of relevance -- until you presented your project that is. The data you have collected is, in my humble opinion, nothing less the a potential treasure trove of information. As such my interest aligns perfectly with the subject and methods of this project.\par

% What have you done before?
Fortunately, my interests are matched by my abilities. Regarding politological subjects I have worked extensively with intra-state conflicts, civil wars and democratic breakdowns. I also have experience with both theories on peace-building and post-conflict reconciliation. Concerning quantitative subjects I have worked with Frequentist econometric, Bayesian statistics, Machine/Deep Learning, geo-spatial analysis, panel data, web-scraping, text analysis, Social Network Analysis, data visualization and Data Science in general. I have acquired my skills and knowledge through courses from the Department of Political Science (UCPH), the Department of Economics (UCPH), the Center of Social Data Science (UCPH) and the Department of Applied Mathematics and Computer Science (DTU) - along with countless hours spend online, looking for more knowledge and new skills. As such, both my theoretical, analytically and methodological foundations are strong and particularly suited for your project.\par % More pol or less quant? 

% more specifically assignment and master thesis
Furthermore I have written both my master's thesis \citep{SPECIALE} and a free-assignment \citep{Maase}  on the subjects of advanced quantitative methods, geo-spatial data and conflict studies. I showed how we can use highly disaggregated to predict future conflicts on a sub-national level \cite{Maase,SPECIALE}. Most recently I presented how we can use advanced machine learning techniques in tandem with data on past conflict patterns to predict the patterns of future conflicts \cite{SPECIALE}. The predictive capabilities of this effort was on par with -- if not better -- then the state of the art in academia. Importantly my results where both highly disaggregated and produced in an setting rigorously emulating a real world forecasting scenario. Challenges which few other previous prediction effort had challenged \citep[48-50]{SPECIALE}. As such I am confident that I can handle the challenges of your project and bring to the effort both relevant insights and experience.\par

% Bridge
Thus, given your focus and your data I see a clear fit between your project on one side and my interest, knowledge and abilities on the other. In the next section I will present a project description outlining my vision of the specific sub-project concerning the correlation between violence in armed conflict and the representation of gender in war photography.\par

\section{Project description} 
%(max. 5 pp. double-spaced, not including bibliography or time schedule) 

[META]

\subsection{The problem}

% ++++++++++++++++++++++++++++++++++++++++++++++++++++++++++++++++++++++++++++++++++++++++++
% THE WORLD; what is the problem?
% ++++++++++++++++++++++++++++++++++++++++++++++++++++++++++++++++++++++++++++++++++++++++++

% Intro 
It has been argued that as violent conflict increases, gender roles changes towards more traditional norms. Men are seen as providers of protection and women as in need of this protection \cite{elshtain1995women, carpenter2003women}. Further more in situations of conflicts it has also been argued that women are perceived as more emotional while men are perceive as more rational \cite{elshtain1993public}. If this is indeed the case then it seems that conflict in itself transform women from actors into subordinate victims -- at least in the eyes of the beholder.\par 

% The more specific problem
This constitutes a number of problems. First of all if violent conflict in itself diminishes the roles of women to that of emotional victims they might lose both their voice and agency doing such conflict \cite{hansen2000gender, hansen2000little}. Naturally this leaves women in a vulnerable and unfavorable position \citep[XXX]{hansen2000little} -- a position which has the potential to linger long after the conflict has settled (ref). Secondly, by reducing the role women to fragile beings in need of male protection we will also fail to recognize the very active - and sometimes destructive - role women have played in conflicts \citep[XXX]{hansen2000gender}. Such blind-spots will impede reconciliation and obscure valuable lesson which could otherwise be used for future conflict prevention.\par

% Why this is very important:
As such, the question of how violent conflicts interacts and changes the perceptions of genders is a highly relevant question not just of academic interest but also of practical importance. Given the data obtained for this project we can now start answering such questions by examining to what extent transformations of gender roles are apparent in the photographs taken by journalists in conflict zones.\par

% Bridge
However, while the project might hold both high relevance and great potential it also face a number of challenges. The use of geo-referenced disaggregated data in conflict studies is still in its infancy, and the addition of image-data even more so. Furthermore past quantitative efforts in conflict studies are experiencing increasing criticism for using fundamentally flawed approach for evaluating statistical results. As such, a myriad of methodological and theological challenges lies before us -- but so does a lot of opportunities. The next subsection will elaborate on the specific challenges while the latter will present the solutions I will use the tackled these challenges.\par


% ++++++++++++++++++++++++++++++++++++++++++++++++++++++++++++++++++++++++++++++++++++++++++
% The Science; what have been done before?
% ++++++++++++++++++++++++++++++++++++++++++++++++++++++++++++++++++++++++++++++++++++++++++
\subsection{The challenges}

%images -- inhibits the questions we can answer...
Naturally, the fact that we are working with journalistic photographs determines the kind of questions we can ask. We can only ask questions regarding how the depiction of gender roles in conflict zones correlates with the level of violence -- not about how the actual gender roles or norms might change in the area. Understanding that the data reflects depiction and presentation rather then experience is crucial when we formulate our hypothesis. Not least since these hypothesis will be the theoretical basis regarding how we are to classify the images.\par 

%images -- Challenges of how to classify the images
This classification of the image data set is it self a huge practical challenge. In practise we do not want to classify whole pictures but specific objects, persons or situations in pictures. Adding to this the scope of the date set, we would be hard pressed to "hand code" all images included in the your data set. We could settle for a subset, but given the rapid development in deep learning techniques for image recognition, we should instead aim to automate as much of the process as possible.\par 

% Why disaggragated data
Turning challenges regarding the the unit of analysis, it is important to recognize that time and space are, at the fundamental level, continues dimensions. However, to work with these dimensions using quantitative tools we need to aggregated them at some discrete level. Most traditional quantitative studies of conflicts have aggregated these dimensions into \emph{country-years} (see such articles as \cite{Collier_Hoeffler_1998, Fearon_Laitin_2003, Collier_Hoeffler_2004, Fearon_2004, Ross_2004, Fearon_2005, Hegre_Sambanis_2006, Goldstone_2010}). While this aggregation might be appropriate in some settings, it does create a very coarse unit of analysis. In reality conflicts -- especially internal -- rarely encompasses entire countries, but are often confined to specific regions \cite[487]{Cederman_Gleditsch_2009}. As such, we miss important nuances and patterns when treating conflict as a phenomenon which is necessarily country-wide. Some regional conflicts will be presented as country-wide, while others will be treated as non-conflicts. As a results past quantitative analyses have often missed important aspects and mechanism of conflicts \cite{Cederman_Gleditsch_2009, Cederman_Gleditsch_Buhaug_2013}. As formulated by \cite{Cederman_Gleditsch_2009}: "If our theories are disaggregated, then our empirical analyses and research designs should reflect this" \citep[490]{Cederman_Gleditsch_2009}. A point which is naturally as relevant for the temporal dimension of conflict as it is for the spatial dimension.\par 

% The challenges of disaagregated data :
Given that your data set contains images with a approximate location and time stamp, and that you will utilize the UCDP data base to track violence in conflicts, we have a great opportunity to create a the specific unit of analysis which best aligns with our theoretical expectation regarding the relationship between the perception of gender roles and violent conflicts. This is crucial, because if we are to ask and answer the right questions we need a theoretical sound unit of analysis, we need to consider our hypothesis when we create this unit and we need to consider the our unit when we handle our data, when we construct our models and when we evaluate our results.\par 

% An example (autocorrolation)
As an example using a highly disaggregated unit of analysis increases the relevance of handle both temporal and spatial auto-correlation. If a violent conflict is present in one village, one specific day it stands to reason that said village will still experience the fallout of such violence the next day, the next week and probably also the next month -- if not even longer depending on the magnitude of the violence. In similar manner the neighboring village might also experience consequences of the violence even if it was not the specific geographical site of the violence. If a journalist where to take a picture the day before the violence, the day of the violence and the day after the violence, we would not expect the images from before and after the violence to be similar, even if they where both taken in the absence of actual violence. The same might be true for pictures taken in the neighboring village. As presented in \cite{SPECIALE} past efforts have tried to ameliorate such problems in various ways, but they have largely failed to create a methodological and theoretical coherent solution for highly disaggregated data.\par 

% An other example (??)
%An other example of how disaggregated data challenges conventional methods and theories is ... \par

% The problem of evaluation
Hvorfor significance er lort...

% bridge
If we are to truly take advantage of the data available for this project and hope to create meaningful and robust insights into the dynamic between gender perception and violent conflict we most take such challenges very seriously. Intriguingly, solving these challenges we will also allows us to create micro level insights regarding not only the correlation between violent conflicts and gender perception but also more specifically examine how gender perceptions diffuses from violent conflicts out through time and space.\par

% ++++++++++++++++++++++++++++++++++++++++++++++++++++++++++++++++++++++++++++++++++++++++++
% This project; How are you gonna do it?
% ++++++++++++++++++++++++++++++++++++++++++++++++++++++++++++++++++++++++++++++++++++++++++

\subsection{The solutions}

%Du siger aggregring skal komme fra teori -> så start med at formulere hypoteser. -> for fortæl hvordna du vil classificere billede jf. de hypoteser -> så fortæl hvrdan du vil agrregare data jf. du hypteser -> så fortæl hvordan du vil håndtere den unit -> så fortæl hvad med så mere kan med GP's -> så fortæl dem hvad svaret på projectet bliver (well elelr hvad du tror.)




% How can you handle the challenges?
To handle the specific challenges of image data I will employ A type of deep learning called \emph{Convoluted Neural Networks} \cite[XXXX]{francois2017deep}. The challenge of creating a framework suitable for disaggregated I will handle by using the Machine learning technique of Gaussian processes \cite{williams2006gaussian}.\par 

% Specifically on images you will....

% This will solve the problem how?

% Specifically on unit of analysis you will....

% This will solve the problem how?

% And we can use the same technique to..

% Spørsmålet er om det her er løsning frem for en udfording..?
%we have a lot, they all need classification. hiring student-coders to classify all images requires a lot of resources ... bla bla... using deep learning for opject reqognition in imarges.... still needs some a large set of hand coded images, but after a model is satisfyingly precise we can add as many new photographs as we want with no need to use scares resources on further trivial coding tasks. But there is an other even more interesting opportunity... which is... [igen læs papire for inspiration]\par


Enquiringly, I have already presented are utilized and approach for handling the spatial 


Appropriately I have already.... bla bla.. Using such modern computational approaches in tandem with the data set containing war photos available for the project we can start answering such questions. bla bla.. \par

% What questions to ask
As formulated in your scholarship-announcement, the overarching research question for the specific sub-project concerns whether there is "[...] a correlation between the level of violence in armed conflict and the representation of gender in war photography?" (ref Fra opslag). Naturally we need even more precise and concrete hypotheses then this.\par % "precise hypothesis"?... Cederman et al. siger noget?


% How to answer these questions?
... \par

% tree cat of photos
But we also have other hypothesis to test; is there any difference between the pictures taken, the pictures send to editing and the pictures published in papers? (what would you expect). Would you expect the publisher to be more prone to fall into stereotypical gender representations then the photographers? Maybe no H, but just but it op as a research area...

% The different genders of the photografers
Furthermore since we have tree male photographers and three female photographers we can also examine. 

% What do you mean by " appropriate and corresponding methodological framework."?
Having these hypothesis we still need to create a methodological framework which will help answer these hypothesis in a theoretical appropriate way.

% Coding of the data
....

% Image recognition
Could perhaps be used, at least we need to try does requirer that we code + 100.000 pictures.

% disaggregation
Using event data such a the data from UCDP is goe-coded for specific coordinates and timestamped for specific time -- with more or less precision. So are the photos (skriv toæ Johan). Naturally these specific coordinates needs to be aggregated into some unite of analysis. Conventional conflict studies have often used country years, but naturally this is much to crude for the project at hand. I purpose a spatial aggregation at the level of the PRIO GRID. (0.5x0.5 decimal degrees). and a temporal aggregation at (mothly?) level. This will give us a unit of analysis consiting of one grid cell one specific month/week. 

% limits of disaggregation
Something about the computational burden of much disaggregation... An that the data can get to sparse..

% Spatial and temporal auto corrolation: GP's
The problem with arbitrary aggregation and for that matter autocorrolation leads to GP's!

% Estimation the "diffusion of change" with GP

% Evaluation -> out-of-sample predicitons/LOO
However, it is not only the estimation framework which is important; so is the evaluation. Out-of-sample prediction or eqavelent is the way forward (greenhill, McEldrith, Shcordt ect) and good arguments.






% OLD: ++++++++++++++++++++++++++++++++++++++++++++++++++++++++++++++++++++++++++++++++++++
% The challenge of auto corrolation
% -------------------

% Better solutions 2 - but we still have a problem
%Two studies which do take a disaggregated approach are \cite{ol2010afghanistan} and \cite{weidmann_ward_2010predicting}. \cite{ol2010afghanistan} do a commendable job of showing the diffusion of conflict from Afghanistan to Pakistan over the Duran Line from a disaggregated perspective, while \cite{weidmann_ward_2010predicting} illustrate how spatial and temporal patterns influenced the civil strife in Bosnia between 1992 and 1995. The challenge is that, using a disaggregated approach, we still have to decide how to include the effects of bad neighbours - the neighbours now simply being some sub-country unit. Indeed the uncertainty of how many bad neighbours to include only amplifies as we move from country to the sub-country level. Here 2$^{nd}$, 3$^{th}$, 4$^{th}$ and potential n$^{th}$ order neighbors will have to be considered, preferably with some demising influence as a function of distance, given the insight from \cite{schutte2011diffusion}. Furthermore, given the bell-curve-like properties of conflict diffusion it seems logical that the magnitude of adjacent conflicts also have an important part to play.\par

%  what we should do
%Unfortunately, both \cite{ol2010afghanistan} and \cite{weidmann_ward_2010predicting} only choose to include a dummy for conflict in 1$^{st}$ order neighbors. The operationalization presented in \cite{Maase} is hardly any better; here distance to nearest conflict is used without taking into account the magnitude of this conflict, or the number of other adjacent conflicts zones. And even if the study had included the 2. nearest conflict, the 3. nearest conflict etc., and scaled the distance by some factor related to the magnitude of conflict, we would still not know if these conflicts surrounded and engulfed the observation of interest or instead clustered neatly and distinctly beside it. The pattern of conflict might matter; the magnitude of conflict might matter; the magnitude of adjacent conflicts in 1$^{st}$, 2$^{nd}$ and n$^{th}$ order might matter. Again the point is clear: modeling features to capture these phenomena must be an estimation effort in and of itself.\par


% --------------------

%Now, using sub-country units requires more computational power, better models, and not least the right data. Such obstacles have previously impeded sub-national analysis on a wider scale. Encouragingly, recent developments in statistics, technology, and data availability address these issues and make highly disaggregated studies possible \citep[446]{ol2010afghanistan}.


Image recognition... can be used for... in time feed into early warning systems where it could amend issues challenges text data ect.


% "The project: “Bodies as Battleground: Gender Images and International Security” addresses three themes: gender, security and images. The project starts from a discursive conception of security, an understanding of gender as cultural, and the image as open to multiple interpretations. The overall research question of the project is: "How are gender-specific security problems constituted through images?" The project employs a mixed-method strategy and is organized around four sub-projects. These show the importance of historical context for how gender security problems might be shown, the relationship between levels of violence and gender representation in war photography, the way gender norms are reproduced or challenge in photographic images, and the possibilities of images to bring theoretical attention to "invisible" security problems."

% The following criteria are used when shortlisting candidates for assessment:

% 1. Research qualifications as reflected in the project proposal.
% 2. Quality and feasibility of the project.
% 3. Qualifications and knowledge in relevant social science disciplines.
% 4. Performance (grades obtained) in graduate and post-graduate studies.

\section{Time schedule} 
...

\section{Temporary bibliography}
% - Temporary
\bibliographystyle{apalike} 
\bibliography{app_conf.bib}

\pagebreak
\section{Appendix}

\subsection{Budget}

\begin{center}
\begin{tabular}{  m{10cm} m{4cm} } 

	\hline
	\textbf{Activity}    & \textbf{Expenditure}\\
	\hline
	    &                               \\
%	Conferences attendances (at least two)   & 6000 - 16.000 Dkr.                       \\
%    Research stay at Oslo or Uppsala (travel ex.)    & 1000 - 2000 Dkr.				\\
%	Cloud computing (rent)   & 1500 - 3000 Dkr.			                                \\
%    Course at DTU (at least two) & 0 Dkr.			                                \\
% 	Activity 5   & X Dkr.	            \\	
%     Activity 6   & X Dkr.			    \\
%     Activity 7   & X Dkr.			    \\
% 	Activity 8   & X Dkr.			    \\
%  	Activity 9   & X Dkr.				\\
 	    &                               \\
 	\hline
    \textbf{Total}       & 8500 - 21.000 Dkr.			    \\
    \hline

\end{tabular}
\end{center}


\pagebreak
\subsection{CV}

My CV can be found in the appended file "\textbf{2\_curriculum\_vitae.pdf}"

%\pagebreak
\subsection{Bachelor's Diploma}

My bachelor's diploma can be found in the appended file "\textbf{3\_diplomas\_and\_transcipts\_of\_records.pdf}"

\subsection{The master's thesis contract and approval}

The master's thesis contract can be found in the appended file "\textbf{3\_diplomas\_and\_transcipts\_of\_records.pdf}"\par

The corresponding approval can be found in the appended file "\textbf{3\_diplomas\_and\_transcipts\_of\_records.pdf}"

\subsection{Grades}

At the time of writing my average at this master's program is 10.8. I finished my bachelor's degree with an average of 7.5. All grades (from master's and bachelor's alike) can be found in the appended file "\textbf{3\_diplomas\_and\_transcipts\_of\_records.pdf}". The official bachelor-grade-card can be found in the append file "\textbf{3\_diplomas\_and\_transcipts\_of\_records.pdf}".\par


\end{document}
