%\pagestyle{article}
\documentclass[a4paper]{article}
\usepackage[english]{babel}
\usepackage[utf8]{inputenc}
\usepackage{graphicx} % for figures
\usepackage[section]{placeins} %This prevents placing floats before a section.
\usepackage{csquotes}

\usepackage{wrapfig} % curtesy of LAURA PETRA - til dit billede i CV

% \usepackage{marginnote} % For margin years CV
% \newcommand{\years}[1]{\marginnote{\scriptsize #1}} % New command for including margin years
% \renewcommand*{\raggedleftmarginnote}{}
% \setlength{\marginparsep}{7pt} % Slightly increase the distance of the margin years from the contant
% \reversemarginpar

\usepackage{natbib}% better citation
%\usepackage{hyperref} %autoref
\usepackage[hidelinks]{hyperref} %hidelinks to remove ugly blue format
\usepackage{amsmath} % for \tag and \eqref macros in mathematical eq.

%% Sets linestretch, paragraphstrech, indentation and footnote stuff 
\usepackage{parskip} %space between paragraphs
\parskip=0pt %set space between paragraphs
\setlength\parindent{12pt} %paragraf indentation
\usepackage[onehalfspacing]{setspace} %linespacing; does not affect footnotes
\setlength{\footnotesep}{0.7\baselineskip}% space between footnotes
\usepackage[hang,flushmargin]{footmisc} %% removes identation in footnoteshttps://www.overleaf.com/project/5b98e00a21d3bd15ac5a2e86

\usepackage{makecell} % make cells in tabels for longer text

\usepackage[colorinlistoftodos]{todonotes}

%% For header and footer (1)
% Marco
\usepackage{fancyhdr}
\pagestyle{fancy}
\textwidth = 452pt% 424pt % test width ish
\oddsidemargin = 0pt %18pt % margin width ish

\fancyheadoffset{0 in} % Shifty solutions..

\fancyhf{} %% clear defuelt header and footer

%% For header and footer (2)
%Specifics
\lhead{Simon P. von der Maase}
\rhead{\today}

\lfoot{University of Copenhagen}
\rfoot{\thepage}
\renewcommand{\footrulewidth}{0.8pt}

\title{\textbf{Bodies as Battleground} \\ PhD Application}

\author{Simon Polichinel von der Maase}
\date{May 2019}

\begin{document}

	\begin{titlepage}
		\maketitle
		%Character count: 43.950/44.000\\
		\noindent\rule{\linewidth}{0.4pt}
		\begin{figure}[h]
			\centering
			\includegraphics[scale=0.32]{KU_logo.png}
		\end{figure}
		\thispagestyle{empty} % removes page number on front page
	\end{titlepage}
    \tableofcontents
\pagebreak

% \section{Application} if below should be subsections.

\section{Motivation and background} 
%Cover Letter detailing your motivation and background for applying for the specific PhD project

% "Are you the right candidate? Preferred applicants will have an interest in questions related to conflict, gender, images, photojournalism and quantitative methodologies. We are looking for candidates with strong analytical skills and proven quantitative methodological qualifications. Experience in working with the Georeferenced Event Dataset from the Uppsala Conflict Data Program or similar datasets is an advantage but not a requirement."

% Du kan overveje at sætte det således op....
%1)interest in questions related to conflict, gender, images, photojournalism and quantitative methodologies. '
%2)We are looking for candidates with strong analytical skills and proven quantitative methodological qualifications. 
%3)Experience in working with the Georeferenced Event Dataset from the Uppsala Conflict Data Program or similar datasets is an advantage but not a requirement."

% intro
As a political scientist with specialization in advanced quantitative methodologies and conflict studies, I was immediately intrigued when I learned of your project. Indeed I see near-perfect alignment between my interest, knowledge and skills on one side an your demands on the other side. The support this assertion, I well here outline my academic background and motivation for applying for this Phd. scholarship.\par

% Why you apply for this?
First of all, conflict research has been my prime politological subject of interest since even before I started my bachelor and it has been one of two focal point doing both my Bachlor's and Master's program. Secondly, my other focal point have been advanced quantitative and computational methods. Indeed it is the intersection between computational methods and conflict studies which have been the fulcrum of throughout my masters program as well as the topic of my Master's thesis. Third, I believe the power to control women's bodies to be a central -- if not the most central -- part of any conflict. As such, I see great potential and importance in pursuing a gender focused approach to conflict studies. Fourth, I have long wanted to use image data in my research, but I have not yet fund any data of relevance -- until you presented your project that is. The data you have collected is, in my humble opinion, nothing less the a potential treasure trove of information. As such my interest aligns rahter perfectly with the subject and methods of this project.\par

% What have you done before?
Fortunately, my interests are matched by my abilities and ambitions. Regarding politological subjects I have worked extensively with intra-state conflicts, civil wars and democratic breakdowns. I also have experience with both theories on peace-building and post-conflict reconciliation. Concerning quantitative subjects I have worked with Frequentist econometric, Bayesian statistics, Machine/Deep Learning, geo-spatial analysis, panel data, web-scraping, automated text analysis, Social Network Analysis, data visualization and Data Science in general. I have acquired these skills through courses from the Department of Political Science (UCPH), the Department of Economics (UCPH), SODAS (UCPH) and the Department of Applied Mathematics and Computer Science (DTU) - along with countless hours spend online, looking for better solutions, more knowledge and new skills. As such, both my theoretical, analytically and methodological foundations are particularly suited for your project.\par % More pol or less quant? 

% more specifically assignment and master thesis
Recently I have written both my master's thesis \citep{SPECIALE} and a free-assignment \citep{Maase} specifically on the subjects of advanced quantitative methods, geo-spatial data and conflict studies. As such I am confident that I can handle the challenges of your project and bring to the effort both relevant insights and experience.\par% In \citep{Maase} I .... In the thesis \citep{SPECIALE} I... 

% Bridge
Thus, given the your focus and your data I see a clear fit between your project on one side and my interest, knowledge and abilities on the other. In the next section I will present a project description outlining my vision of the specific sub-project concerning the correlation between violence in armed conflict and the representation of gender in war photography.\par

% 1/5 page left for cover letter! 

\section{Project description} 
%(max. 5 pp. double-spaced, not including bibliography) 

% THE WORLD; what is the problem? ++++++++++++++++++++++++++++++++++++++++++++++++++++
% Intro 
It has been argued that as violent conflict increases, gender roles moves towards more traditional norms. Men are seen as providers of protection and women as in need of this protection (Elshtain 1987, Carpenter 2003: LÆS!). Further more in situations of conflicts it has also been argued that women are perceived as more emotional while men are perceive as more rational (Elshtain 1981, Tickner 2007 LÆS!). If this is indeed the case then it seems that conflict in itself transform women from actors into subordinate victims -- at least in the eyes of the beholder. 

% The more specific problem
This constitutes a number of problems. First of all if violent conflict in itself diminishes the roles of women to that of emotional victims they might lose both their voice and agency doing such conflict (Lene1 og Lene2). Naturally this leaves women in a very vulnerable and unfavorable position (Lene2)-- a position which has the potential to linger long after the conflict has settled (ref). Secondly, by reducing the role women to fragile beings in need of male protection we will also fail to recognize the very active - and sometimes destructive - role women have played in conflicts (spartan mothers. ref: Lene1). Such blind-spots will impede reconciliation and obscure valuable lesson which could otherwise be used for future conflict prevention.\par

% Why this is very important:
As such, the question of how violent conflicts interacts and changes the perceptions of genders is an highly relevant question. Given the data obtained for this project we can now start answering such question by examining to what extent transformations of gender roles are apparent in the photographs taken by journalists in conflict zones .\par

% How are gender-specific security problems constituted through images.
% Lack of systematic studies of how images represent gender specific security problems...
% Bodies as battleground... read background lit... Women's bodies are contested... 
% Gendeer based ... not been incorporate into secutity theories

% Bridge
However, while the project might hold both high relevance and great potential, using images as data in quantitative frameworks is a very new addition to social sciences research -- even more so conflict studies specifically. [læs papir om image data]. As such, a myriad of methodological and theological challenges lies before us -- and so does a lot of opportunities.\par

% The Science; how have this problem been handled before? +++++++++++++++++++++++++++++
% Specific challenges in light of current research
Specifically, two major challenges facing the effort is how to best utilize the image data and how to handle the unit of analysis in a theoretical and methodological sound manner.\par

% with imarges: ()
- images -- need to answer what you will use deep learning for?

% Why disaggragated data
Regarding the unit of analysis, most traditional quantitative studies of conflicts have used country-years \cite{XXX}. This is naturally A very coarse unit of analysis. In reality conflicts - especially internal - rarely encompasses entire countries, but are often confined to specific regions \cite[487]{Cederman_Gleditsch_2009}. As such, we miss important nuances and patterns when treating conflict as a phenomenon which is necessarily country-wide. Some regional conflicts will be presented as country-wide, while other serious regional conflicts will be treated as non-conflicts. As a results past quantitative analyses have often missed important aspects and mechanism of conflicts \cite{Cederman_Gleditsch_2009, Cederman_Gleditsch_Buhaug_2013}. As formulated by \cite{Cederman_Gleditsch_2009}: "If our theories are disaggregated, then our empirical analyses and research designs should reflect this" \citep[490]{Cederman_Gleditsch_2009}.\par 

% The challenges of disaagregated data (autocorrolation):
Importantly this point is just as relevant for the temporal dimension of conflict as it is for the spatial dimension. That appropriate unit of analysis might well be both sub-country and sub-year. We need a sound unit of analysis and we need to consider this unit of analysis when we formulate our hypothesis, when we choose our data, When we construct our models and when we evaluate our results. as an example using a highly disaggregated unit of analysis increases the relevance of handle both temporal and spatial auto-correlation. If a violent conflict is present in one village, one specific day it stands to reason that said village will still experience the fallout of said violence the next day, week and month even if the violence stops. In similar manner the neighboring village might also experience consequences of the violence even if it was not the specific geographical site of the conflict. As presented in \cite{SPECIALE} past efforts have largely failed to create a methodological and theoretical coherent solution to this problem.\par 

% Other challenges with disaggragated data.


% bridge
Regarding 

- the cool thing about GP; you can estimate the spread in time and space - if we find a corrolation...

% This project; How are you gonna do it (new/better)? ++++++++++++++++++++++++++++++
% How are you gonna handle these challenges?
Using the data set containing war photos available for the project in tandem with modern computational methods we can start answering such questions.

% overall RC
As formulated in the Phd scholarship annoncement, the overarching research question for the specific sub-project is whether there
"[...] a correlation between the level of violence in armed conflict and the representation of gender in war photography?" (ref Fra opslag)

% More speicifically
% The literature suggests that more violence leads more traditional gender norms (ref. from Lenes app.). Unfortunately we can not deduced the actually genders norms and roles, but we can deduced how genders are portrait and represented in war photography.

% How?
We have good data. What we need now is precise hypothesis with clear theoretical underpinnings and a appropriate and corresponding methodological framework....

% What do you mean by "precise hypothesis"?
... Cederman et al. siger noget?

% tree cat of photos
but we also have other hypothesis to test; is there any difference between the pictures taken, the pictures send to editing and the pictures published in papers? (what would you expect). Would you expect the publisher to be more prone to fall into stereotypical gender representations then the photographers? Maybe no H, but just but it op as a research area...

% The different genders of the photografers
Furthermore since we have tree male photographers and three female photographers we can also examine. 

% What do you mean by " appropriate and corresponding methodological framework."?
Having these hypothesis we still need to create a methodological framework which will help answer these hypothesis in a theoretical appropriate way.

% Coding of the data
....

% Image recognition
Could perhaps be used, at least we need to try does requirer that we code + 100.000 pictures.

% Agrregation
Using event data such a the data from UCDP is goe-coded for specific coordinates and timestamped for specific time -- with more or less precision. So are the photos (skriv toæ Johan). Naturally these specific coordinates needs to be aggregated into some unite of analysis. Conventional conflict studies have often used country years, but naturally this is much to crude for the project at hand. I purpose a spatial aggregation at the level of the PRIO GRID. (0.5x0.5 decimal degrees). and a temporal aggregation at (mothly?) level. This will give us a unit of analysis consiting of one grid cell one specific month/week. 

% limits of disaggregation
Something about the computational burden of much disaggragation... An that the data can get to sparse..

% Spatial and temporal auto corrolation: GP's
The problem with arbitrary aggregation and for that matter autocorrolation leads to GP's!


% Evaluation -> out-of-sample predicitons/LOO
However, it is not only the estimation framework which is important; so is the evaluation. Out-of-sample prediction or eqavelent is the way forward (greenhill, McEldrith, Shcordt ect) and good arguments.






% OLD: ++++++++++++++++++++++++++++++++++++++++++++++++++++++++++++++++++++++++++++++++++++
% The challenge of auto corrolation
% -------------------

% Better solutions 2 - but we still have a problem
%Two studies which do take a disaggregated approach are \cite{ol2010afghanistan} and \cite{weidmann_ward_2010predicting}. \cite{ol2010afghanistan} do a commendable job of showing the diffusion of conflict from Afghanistan to Pakistan over the Duran Line from a disaggregated perspective, while \cite{weidmann_ward_2010predicting} illustrate how spatial and temporal patterns influenced the civil strife in Bosnia between 1992 and 1995. The challenge is that, using a disaggregated approach, we still have to decide how to include the effects of bad neighbours - the neighbours now simply being some sub-country unit. Indeed the uncertainty of how many bad neighbours to include only amplifies as we move from country to the sub-country level. Here 2$^{nd}$, 3$^{th}$, 4$^{th}$ and potential n$^{th}$ order neighbors will have to be considered, preferably with some demising influence as a function of distance, given the insight from \cite{schutte2011diffusion}. Furthermore, given the bell-curve-like properties of conflict diffusion it seems logical that the magnitude of adjacent conflicts also have an important part to play.\par

%  what we should do
%Unfortunately, both \cite{ol2010afghanistan} and \cite{weidmann_ward_2010predicting} only choose to include a dummy for conflict in 1$^{st}$ order neighbors. The operationalization presented in \cite{Maase} is hardly any better; here distance to nearest conflict is used without taking into account the magnitude of this conflict, or the number of other adjacent conflicts zones. And even if the study had included the 2. nearest conflict, the 3. nearest conflict etc., and scaled the distance by some factor related to the magnitude of conflict, we would still not know if these conflicts surrounded and engulfed the observation of interest or instead clustered neatly and distinctly beside it. The pattern of conflict might matter; the magnitude of conflict might matter; the magnitude of adjacent conflicts in 1$^{st}$, 2$^{nd}$ and n$^{th}$ order might matter. Again the point is clear: modeling features to capture these phenomena must be an estimation effort in and of itself.\par


% --------------------

%Now, using sub-country units requires more computational power, better models, and not least the right data. Such obstacles have previously impeded sub-national analysis on a wider scale. Encouragingly, recent developments in statistics, technology, and data availability address these issues and make highly disaggregated studies possible \citep[446]{ol2010afghanistan}.


Image recognition... can be used for... in time feed into early warning systems where it could amend issues challenges text data ect.


% "The project: “Bodies as Battleground: Gender Images and International Security” addresses three themes: gender, security and images. The project starts from a discursive conception of security, an understanding of gender as cultural, and the image as open to multiple interpretations. The overall research question of the project is: "How are gender-specific security problems constituted through images?" The project employs a mixed-method strategy and is organized around four sub-projects. These show the importance of historical context for how gender security problems might be shown, the relationship between levels of violence and gender representation in war photography, the way gender norms are reproduced or challenge in photographic images, and the possibilities of images to bring theoretical attention to "invisible" security problems."

% The following criteria are used when shortlisting candidates for assessment:

% 1. Research qualifications as reflected in the project proposal.
% 2. Quality and feasibility of the project.
% 3. Qualifications and knowledge in relevant social science disciplines.
% 4. Performance (grades obtained) in graduate and post-graduate studies.

\section{Time schedule} 
...

\subsection{Temporary bibliography}
% - Temporary
\bibliographystyle{apalike} 
\bibliography{app_conf.bib}

\pagebreak
\section{Appendix}

\subsection{Budget}

\begin{center}
\begin{tabular}{  m{10cm} m{4cm} } 

	\hline
	\textbf{Activity}    & \textbf{Expenditure}\\
	\hline
	    &                               \\
%	Conferences attendances (at least two)   & 6000 - 16.000 Dkr.                       \\
%    Research stay at Oslo or Uppsala (travel ex.)    & 1000 - 2000 Dkr.				\\
%	Cloud computing (rent)   & 1500 - 3000 Dkr.			                                \\
%    Course at DTU (at least two) & 0 Dkr.			                                \\
% 	Activity 5   & X Dkr.	            \\	
%     Activity 6   & X Dkr.			    \\
%     Activity 7   & X Dkr.			    \\
% 	Activity 8   & X Dkr.			    \\
%  	Activity 9   & X Dkr.				\\
 	    &                               \\
 	\hline
    \textbf{Total}       & 8500 - 21.000 Dkr.			    \\
    \hline

\end{tabular}
\end{center}


\pagebreak
\subsection{CV}

My CV can be found in the appended file "\textbf{2\_curriculum\_vitae.pdf}"

%\pagebreak
\subsection{Bachelor's Diploma}

My bachelor's diploma can be found in the appended file "\textbf{3\_diplomas\_and\_transcipts\_of\_records.pdf}"

\subsection{The master's thesis contract and approval}

The master's thesis contract can be found in the appended file "\textbf{3\_diplomas\_and\_transcipts\_of\_records.pdf}"\par

The corresponding approval can be found in the appended file "\textbf{3\_diplomas\_and\_transcipts\_of\_records.pdf}"

\subsection{Grades}

At the time of writing my average at this master's program is 10.8. I finished my bachelor's degree with an average of 7.5. All grades (from master's and bachelor's alike) can be found in the appended file "\textbf{3\_diplomas\_and\_transcipts\_of\_records.pdf}". The official bachelor-grade-card can be found in the append file "\textbf{3\_diplomas\_and\_transcipts\_of\_records.pdf}".\par


\end{document}
